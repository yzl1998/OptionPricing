\documentclass[11pt, oneside]{article}   	% use "amsart" instead of "article" for AMSLaTeX format
\usepackage{geometry}                		% See geometry.pdf to learn the layout options. There are lots.
\geometry{letterpaper}                   		% ... or a4paper or a5paper or ... 
%\geometry{landscape}                		% Activate for rotated page geometry
%\usepackage[parfill]{parskip}    		% Activate to begin paragraphs with an empty line rather than an indent
\usepackage{graphicx}				% Use pdf, png, jpg, or eps§ with pdflatex; use eps in DVI mode
								% TeX will automatically convert eps --> pdf in pdflatex		
\usepackage{amssymb}
\usepackage[ruled,vlined]{algorithm2e}
\SetKwFor{For}{for (}{) $\lbrace$}{$\rbrace$}
\usepackage{listings}
\usepackage{url}


%SetFonts

%SetFonts


\title{ Report: Least Squares Monte Carlo Python Implementation}
\author{Zhonglin Yang}
\date{}							% Activate to display a given date or no date

\begin{document}
\maketitle 
\noindent The report presents a Python implementation of Longstaff and Schwartz's least squares Monte Carlo approach of valuing American options by simulation 
(from \cite{10.1093/rfs/14.1.113}).
\section{Longstaff and Schwartz's Method}

\subsection{Monte-Carlo simulation}
One of the most common ways to value financial products such as options is using Monte Carlo simulation. The basic idea of the approach is creating a large sample of paths of random processes for the underlying asset such stock prices, and then explicitly computing the expected value of the discounted payoff using these paths.\subsection{Least squares regression}
Longstaff and Schwartz proposed a least squares Monte Carlo approach to evaluate American-style options in 2001. To understand thee intuition behind the approach, we need to recall that the holder of an American option optimally compare the immediate exercise payoff with the expected continuation payoff at any exercise time, and choose to exercise the option only if the immediate exercise payoff is higher than the expected continuation payoff. The key point to evaluate American options by Monte Carlo simulation is to be able to approximate the conditional expectation at any exercise time. Longstaff and Schwartz proposed to use least squares regression to directly find the conditional expectation function at each exercise time. By using the conditional expectation functions, a complete optimal exercise strategy can be found along each path. And with the complete optimal exercise strategy, American options can be evaluated accurately. 
\subsection{High level of practical implementation}
Several articles also addressed valuing American options by simulation around the time that Longstaff and Schwartz published their article. However, none of these articles introduced an approach as practical as Longstaff and Schwartz's approach. Longstaff and Schwartz only include paths for which the option is in the money in doing least squares regression. It increases the efficiency of the approach.

\section{LSM Approach Algorithm}
\subsection{The overall algorithm}
The overall algorithm for Longstaff and Schwartz's least squares Monte Carlo approach is given as follows.\newline
\begin{algorithm}[H]
\SetAlgoLined
generate stock price paths(m, n)\;
 \For{$i = n;\ i > 0;\ i --$}{
do least square regressions($i$)\; 
determine cashflows for each path($i$)\;
}
calculate the option price\;
\caption{Valuing American Options by LSM approach}
\end{algorithm}

\subsection{Generate stock price paths}
To simulate stock prices and generate paths, we suggest that stock prices follow a geometric Brownian motion.
\begin{equation}
dS_t = rS_tdt + \sigma S_tdW_t
\end{equation}
Thus, we have $\Delta S = Srdt + S\sigma dW \sqrt[]{dt}$ where dW is the random variable which follows a normal distribution. We random generates $m \times n$ random vairables $dW$, create a $m \times n$ matrix $S$, and store $S[i, t]$ for all $i \in [1,m] $ and all $t \in [1,n] $where $S[i, t]$ represents the stock price of the $i$th path at timestamp $t$.

\subsection{Do least squares regression}
We use the first seven Legendre polynomial functions as basis functions for least squares regression and they are $L_0(X), L_1(X)$ ... $L_6(X)$.\newline
\newline
When we have generate $m$ stock price paths, we can do least squares regression for each timestamp $t$ for all $t \in [1, n]$ backwards. Suppose we want to do least squares regression at timestamp $k$, since we already know $S[i, k]$ for all $i \in [1, m]$, we can collect all the pairs of $S[i, k]$ and discounted $cashflow[i, k+1]$ whose $S[i, k]$ is in the money. Then, Suppose $l$ pairs are collected, according to the $l$ pairs values, we can construct a $l\times7$ martix $\phi$ and a $l \times 1$ matrix $Y$ where $Y$ is filled with the $l$ discounted $cashflow$ values and each column $\phi$ stores the corresponding $S$ values' Legendre polynomials. For example, the first pair is $(S_1, cashflow_1)$, then $Y[1]$ stores $cashflow_1$ and $\phi[1]$ is $[L_0(S_1),L_1(S_1)$ ... $,L_6(S_1)]$. Next, we construct $A = \phi^T \cdot \phi$, and $b = \phi^T \cdot Y$. Lastly, we solve the linear system $Ax = b$. $x$ stores seven coefficients  $x[0], x[1]$ ... $,x[6]$. Then the conditional expectation function $E(X)$ is $x[0] \cdot L_0(X) + x[1] \cdot L_1(X)  $ ... $ + x[6] \cdot L_6(X)$.

\subsection{Determine cashflows and evaluate the American option}
After getting the conditional expectation functions, we can determine whether exercise the American option at each timestamp or not. For $S[i, t]$, if $payoff(S[i, t])$ < $E(S[i,t])$, then we shouldn't exercise the option at timestamp $t$ in the $i$th path. After determining the optimal exercise time for each path, the value of the American option can be computed based on simulation results. It is \[  \sum_{n=1}^{m} payoff(S[n, opt(n)]) \cdot discount(opt(n)).\]


\section{Python Implementation Code}
\subsection{Legendre polynomials}
We choose the first seven Laguerre polynomials as basis functions for least-squares regression. And to handle the extreme behaviors, we multiply each polynomial by an additional exponential term. The following code snippet is the implementation of Legendre polynomial functions.
\lstset{language=Python}
\lstset{frame=lines}
\lstset{label={lst:code_direct}}
\lstset{basicstyle=\footnotesize}
\begin{lstlisting}
import math

def l_0(val_arg):
    return math.exp(-val_arg/2)

def l_1(val_arg):
    return math.exp(-val_arg/2)*(1-val_arg)

def l_2(val_arg):
    return math.exp(-val_arg/2)*(1-2*val_arg+1/2*(val_arg**2))

def l_3(val_arg):
    return math.exp(-val_arg/2)*(1-3*val_arg+3/2*(val_arg**2)-1/6*(val_arg ** 3))

def l_4(val_arg):
    return math.exp(-val_arg/2)*(1-4*val_arg+3*(val_arg**2)
                                -3/4*(val_arg ** 3)+1/24*(val_arg**4))

def l_5(val_arg):
    return math.exp(-val_arg/2)*(1-5*val_arg+5*(val_arg**2)-10/6*(val_arg**3)
                               + 25/120*(val_arg**4)-1/120*(val_arg**5))

def l_6(val_arg):
    return math.exp(-val_arg/2)*(1-6*val_arg+540/72*(val_arg**2)
                                  -240/72*(val_arg**3)+45/72*(val_arg**4)
                                  -1/20*(val_arg**5)+1/720*(val_arg**6))
\end{lstlisting}


\subsection{Determination of cashflow}
We use three helper functions to decide whether the American option should be exercised or not at a given timestamp.
\lstset{language=Python}
\lstset{frame=lines}
\lstset{label={lst:code_direct}}
\lstset{basicstyle=\footnotesize}
\begin{lstlisting}
# check whether to exercise or continue
def whether_exercise(S,K,k,coefficients,test_functions):
    continue_val = 0
    n = len(test_functions)
    for i in range(0, n):
        continue_val+ = coefficients[k][i]*test_functions[i](S)
    return max(K - S,0.0) > continue_val
    
# evaluate the payoff
def payoff(exercise_val,path,i,K,m):
    for j in range(0, m):
        exercise_val[j] = max(0.0,K-path[j][i])
        
# compare the continuation value and the current payoff
def determine(continue_val,exercise_val,cashflow,m):
    for i in range(0, m):
        if continue_val[i] < exercise_val[i]:
            cashflow[i] = exercise_val[i]
\end{lstlisting}


\subsection{Least-squares regression}
The following code snippet is the least-squares regression function where the coefficients are calculated according to the given basis functions.
\lstset{language=Python}
\lstset{frame=lines}
\lstset{label={lst:code_direct}}
\lstset{basicstyle=\footnotesize}
\begin{lstlisting}
def regression(x_regression,y_regression,size_regression,coefficients,k,test_functions):
    size_1 = size_regression
    size_2 = len(test_functions)
    y_regression_2 = np.zeros(size_regression)

    # construct the matrix phi
    d_phi = np.zeros((size_1, size_2))
    for i in range(0, size_1):
        y_regression_2[i] = y_regression[i]
        for j in range(0, size_2):
            d_phi[i][j] = test_functions[j](x_regression[i])

    # construct A = phi^T*phi, b = phi^T*Y
    d_phi_transpose = d_phi.transpose()
    A = np.dot(d_phi_transpose, d_phi)
    b = np.dot(d_phi_transpose, y_regression_2)

    # solve linear equations Ax = b
    try:
        x = np.linalg.solve(A, b)
        for i in range(0, size_2):
            coefficients[k][i] = x[i]
    except np.linalg.LinAlgError as err:
        raise np.linalg.LinAlgError()
\end{lstlisting}


\subsection{Projection of continuation value}
The projection function is used to evaluate continue values of each path at a given timestamp. The coefficients are calculated by the least-squares regression function.
\lstset{language=Python}
\lstset{frame=lines}
\lstset{label={lst:code_direct}}
\lstset{basicstyle=\footnotesize}
\begin{lstlisting}
def projection(continue_val,coefficients,k,test_functions,path,m):
    n = len(test_functions)
    for i in range(0, m):
        continue_val[i] = 0
        for j in range(0, n):
            continue_val[i]+ = coefficients[k][j] * test_functions[j](path[i][k])
\end{lstlisting}


\subsection{Path generations and regression steps}
In the euler\underline{ }path function, firstly, all pilot paths are generated. Then the regression steps are done backwards. Lastly, according to the results from the regression steps, exercise boundaries are determined and the value of the American option are calculated.
\lstset{language=Python}
\lstset{frame=lines}
\lstset{label={lst:code_direct}}
\lstset{basicstyle=\footnotesize}
\begin{lstlisting}
def euler_path(m,n,S0,sigma,r,K,T,dW,coefficients,test_functions):
    path = np.zeros((m, n))
    dt = T/n
    S = S0

    # generate all paths
    for j in range(0, m):
        for i in range(0, n):
            S+ = r*S*dt + sigma[i]*S*math.sqrt(dt)*dW[j][i]
            path[j][i] = S
        S = S0

    # do the regression steps backward
    continue_val = np.zeros(m)
    exercise_val = np.zeros(m)
    cashflow = np.zeros(m)

    for i in range(n - 1, 0, -1):
        projection(continue_val,coefficients,i,test_functions,path,m)
        payoff(exercise_val,path,i,K,m)
        determine(continue_val,exercise_val,cashflow,m)

        x_regression = np.zeros(m)
        y_regression = np.zeros(m)
        size_regression = 0

        for j in range(0, m):
            cashflow[j]* = math.exp(-r*dt)
            if path[j][i] < K:
                x_regression[size_regression] = path[j][i - 1]
                y_regression[size_regression] = cashflow[j]
                size_regression+ = 1

        regression(x_regression,y_regression,size_regression,
        	   coefficients,i - 1,test_functions)

    projection(continue_val,coefficients,0,test_functions,path,m)
    payoff(exercise_val, path,0,K,m)
    determine(continue_val,exercise_val,cashflow,m)
    V = 0
    
    # evaluate the option
    for j in range(0, m):
        V+ = math.exp(-r*dt) * cashflow[j]
        
    return V/m
\end{lstlisting}


\subsection{American option class and tangent method for pricing}
To evaluate the American option by using the tangent method. We create an AmericanOption Class and define the american\underline{ }price function where the value and greek letters of the American option are calculated by using tangent method.
\lstset{language=Python}
\lstset{frame=lines}
\lstset{label={lst:code_direct}}
\lstset{basicstyle=\footnotesize}
\begin{lstlisting}
class AmericanOption(object):

    def __init__(self,m,n):
        self.m = m
        self.n = n
        self.S0 = 1
        self.T = 1
        self.K = 1
        self.r = 0.04
        self.sigma = np.full(n, 0.2)
        self.vega = np.full(n, 0)
        self.vanna = np.full(n, 0)
        self.dW = np.zeros((m, n))
        self.S = self.S0
        self.delta = 1
        self.gamma = 0
        self.vegaBS = 0
        self.vannaBS = 0
        self.V = 0

    # the pricing function using tangent method
    def american_price(self,coefficients,test_functions):
        sum, sumt, sumd, sumtt, sumtd = 0, 0, 0, 0, 0
        S0 = self.S
        delta0, vegaBS0, gamma0, vannaBS0 = self.delta, self.vegaBS, self.gamma,\
        			 	    self.vannaBS
        dt = self.T/self.n

        for j in range(0, self.m):
            t = 0
            for i in range(0, self.n):
                self.vannaBS += self.r*self.vannaBS*dt\
                		+ self.sigma[i]*self.vannaBS*math.sqrt(dt)*self.dW[j][i]\
                                + self.delta*math.sqrt(dt)*self.dW[j][i]
                self.vegaBS += self.r*self.vegaBS*dt\ 
                		+ self.sigma[i]*self.vegaBS*math.sqrt(dt)*self.dW[j][i]\
                                + self.S*math.sqrt(dt)*self.dW[j][i]
                self.delta += self.r*self.delta*dt\
                		+ self.sigma[i]*self.delta*math.sqrt(dt)*self.dW[j][i]
                self.S += self.r*self.S*dt\
                		+ self.sigma[i]*self.S*math.sqrt(dt)*self.dW[j][i]
                t = dt*(i+1)
                if whether_exercise(self.S, self.K, i, coefficients, test_functions):
                    sum+ = max(self.K - self.S, 0.0)*math.exp(- self.r*t)
                    sumt+ = smooth_digit(self.S, self.K)*math.exp(- self.r*t)*self.delta
                    sumd+ = smooth_digit(self.S, self.K)*math.exp(- self.r*t)*self.vegaBS
                    sumtt+ = (self.delta**2)*math.exp(-self.r*self.T)\
                    	     *smooth_diac(self.S,self.K)
                    sumtd+ = math.exp(-self.r*t)*smooth_diac(self.S, self.K)*self.delta\
                    	     *self.vegaBS + smooth_digit(self.S, self.K)\
	     	             *math.exp(-self.r*t)*self.vannaBS
                    break

            self.S = S0
            self.delta = delta0
            self.gamma = gamma0
            self.vegaBS = vegaBS0
            self.vannaBS = vannaBS0

        self.S = sum/self.m
        self.delta = sumt/self.m
        self.gamma = sumtt/self.m
        self.vegaBS = sumd/self.m
        self.vannaBS = sumtd/self.m
\end{lstlisting}


\subsection{Main function}
Lastly, we define the main function to initialize an AmericanOption object and use its attributes to call euler\underline{ }path function so that we can get the coefficients of basis functions. Then, we pass the coefficients to the american\underline{ }price function to calculate the price and greek letters of the American option.
\lstset{language=Python}
\lstset{frame=lines}
\lstset{label={lst:code_direct}}
\lstset{basicstyle=\footnotesize}
\begin{lstlisting}
def main(m, n):
    ao = AmericanOption(m, n)

    np.random.seed(0)
    ao.dW = np.random.normal(0,1,(m,n))

    test_functions = [l_0, l_1, l_2, l_3, l_4, l_5, l_6]
    coefficients = np.full((n, len(test_functions)), 0.0)

    # generate the paths and calculate the regression coefficients
    ao.V = euler_path(m,n,ao.S0,ao.sigma,ao.r,ao.K,ao.T,ao.dW,\
    		      coefficients,test_functions)

    print(coefficients)

    np.random.seed(1)
    ao = AmericanOption(2*m, n)
    ao.dW = np.random.normal(0,1,(2*m,n))

    # calculate the price of the American option
    ao.american_price(coefficients,test_functions)

    # print out the calculated price and greek letters of the American option
    print(`V = {}'.format(ao.V))
    print(`S = {}'.format(ao.S))
    print(`delta = {}'.format(ao.delta))
    print(`gamma = {}'.format(ao.gamma))
    print(`vegaBS = {}'.format(ao.vegaBS))
    print(`vannaBS = {}'.format(ao.vannaBS))


t0 = time()
main(100000, 100)
t1 = time()
d1 = t1 - t0
print (d1)
\end{lstlisting}

\subsection{Miscellaneous functions}
Two approximation functions which are used in the american\underline{ }price function. 
\lstset{language=Python}
\lstset{frame=lines}
\lstset{label={lst:code_direct}}
\lstset{basicstyle=\footnotesize}
\begin{lstlisting}
def smooth_digit(S, K):
    return -1.0 / (1.0 + math.exp(-(K - S) / 0.005))

def smooth_diac(S, K):
    return math.exp(-(K-S)/0.005)/0.005/
    		    ((1.0+math.exp(-(K-S)/0.005))*(1.0+math.exp(-(K-S)/0.005)))
\end{lstlisting}

\section{Numerical Results}
With $S_0= 1,  r = 0.04,  \sigma = 0.2, T = 1,$ and $K=1$, Geeske and Johnson got the analytical results (price, delta, vega, gamma, and vanna) of the American option in \cite{https://doi.org/10.1111/j.1540-6261.1984.tb04921.x}.
Pass the same attributes with $ m = 100,000, n =100$, and three various random variable seeds to our program, we get our tangent method results and compare our results with the analytical results in the following table.
\begin{center}
 \begin{tabular}{||c c c c c||} 
 \hline
  & Analytical Results& Seed = 0 & Seed = 1 &Seed = 2 \\ [0.5ex] 
 \hline\hline
 Price & 0.0640 &0.05969&0.05925& 0.0630\\ 
 \hline
 Delta & -0.4160 & -0.3810 &-0.3685 &  -0.4069\\
 \hline
 Vega &0.3833 & 0.3472 &  0.3463&0.3719\\
 \hline
 Gamma & 1.1069 & 0.4947 &0.5064 & 0.5853 \\
 \hline
 Vanna & 0.4247 & 0.1837 & 0.1793 & 0.1787\\ [1ex] 
 \hline
\end{tabular}
\end{center}

\bibliographystyle{unsrt}
\bibliography{mybib}
\end{document}  